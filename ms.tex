 %%
%% Beginning of file 'sample62.tex'
%%
%% Modified 2018 January
%%
%% This is a sample manuscript marked up using the
%% AASTeX v6.2 LaTeX 2e macros.
%%
%% AASTeX is now based on Alexey Vikhlinin's emulateapj.cls 
%% (Copyright 2000-2015).  See the classfile for details.

%% AASTeX requires revtex4-1.cls (http://publish.aps.org/revtex4/) and
%% other external packages (latexsym, graphicx, amssymb, longtable, and epsf).
%% All of these external packages should already be present in the modern TeX 
%% distributions.  If not they can also be obtained at www.ctan.org.

%% The first piece of markup in an AASTeX v6.x document is the \documentclass
%% command. LaTeX will ignore any data that comes before this command. The 
%% documentclass can take an optional argument to modify the output style.
%% The command below calls the preprint style  which will produce a tightly 
%% typeset, one-column, single-spaced document.  It is the default and thus
%% does not need to be explicitly stated.
%%
%%
%% using aastex version 6.2
\documentclass{aastex62}

%% The default is a single spaced, 10 point font, single spaced article.
%% There are 5 other style options available via an optional argument. They
%% can be envoked like this:
%%
%% \documentclass[argument]{aastex62}
%% 
%% where the layout options are:
%%
%%  twocolumn   : two text columns, 10 point font, single spaced article.
%%                This is the most compact and represent the final published
%%                derived PDF copy of the accepted manuscript from the publisher
%%  manuscript  : one text column, 12 point font, double spaced article.
%%  preprint    : one text column, 12 point font, single spaced article.  
%%  preprint2   : two text columns, 12 point font, single spaced article.
%%  modern      : a stylish, single text column, 12 point font, article with
%% 		  wider left and right margins. This uses the Daniel
%% 		  Foreman-Mackey and David Hogg design.
%%  RNAAS       : Preferred style for Research Notes which are by design 
%%                lacking an abstract and brief. DO NOT use \begin{abstract}
%%                and \end{abstract} with this style.
%%
%% Note that you can submit to the AAS Journals in any of these 6 styles.
%%
%% There are other optional arguments one can envoke to allow other stylistic
%% actions. The available options are:
%%
%%  astrosymb    : Loads Astrosymb font and define \astrocommands. 
%%  tighten      : Makes baselineskip slightly smaller, only works with 
%%                 the twocolumn substyle.
%%  times        : uses times font instead of the default
%%  linenumbers  : turn on lineno package.
%%  trackchanges : required to see the revision mark up and print its output
%%  longauthor   : Do not use the more compressed footnote style (default) for 
%%                 the author/collaboration/affiliations. Instead print all
%%                 affiliation information after each name. Creates a much
%%                 long author list but may be desirable for short author papers
%%
%% these can be used in any combination, e.g.
%%
%% \documentclass[twocolumn,linenumbers,trackchanges]{aastex62}
%%
%% AASTeX v6.* now includes \hyperref support. While we have built in specific
%% defaults into the classfile you can manually override them with the
%% \hypersetup command. For example,
%%
%%\hypersetup{linkcolor=red,citecolor=green,filecolor=cyan,urlcolor=magenta}
%%
%% will change the color of the internal links to red, the links to the
%% bibliography to green, the file links to cyan, and the external links to
%% magenta. Additional information on \hyperref options can be found here:
%% https://www.tug.org/applications/hyperref/manual.html#x1-40003
%%
%% If you want to create your own macros, you can do so
%% using \newcommand. Your macros should appear before
%% the \begin{document} command.
%%

\usepackage{graphicx}
\usepackage{float}
\usepackage[caption=false]{subfig}
\usepackage{enumitem}

%\usepackage{natbib}
%\usepackage{pdflscape}

%% AASTeX v6.* now includes \hyperref support. While we have built in specific
%% defaults into the classfile you can manually override them with the
%% \hypersetup command. For example,
%%
%%\hypersetup{linkcolor=red,citecolor=green,filecolor=cyan,urlcolor=magenta}
%%
%% will change the color of the internal links to red, the links to the
%% bibliography to green, the file links to cyan, and the external links to
%% magenta. Additional information on \hyperref options can be found here:
%% https://www.tug.org/applications/hyperref/manual.html#x1-40003

%% If you want to create your own macros, you can do so
%% using \newcommand. Your macros should appear before
%% the \begin{document} command.
%%
\newcommand{\vdag}{(v)^\dagger}
\newcommand\aastex{AAS\TeX}
\newcommand\latex{La\TeX}
\newcommand{\exampleConstant}{0.04}
\newcommand{\degree}{^\circ}
\newcommand{\spitzer}{{\it Spitzer}}
\newcommand{\kepler}{{\it Kepler}}

%% Reintroduced the \received and \accepted commands from AASTeX v5.2
%\received{July 1, 2016}
%\revised{September 27, 2016}
%\accepted{\today}
%% Command to document which AAS Journal the manuscript was submitted to.
%% Adds "Submitted to " the arguement.
%\submitjournal{ApJ}

%% Mark up commands to limit the number of authors on the front page.
%% Note that in AASTeX v6.1 a \collaboration call (see below) counts as
%% an author in this case.
%
%\AuthorCollaborationLimit=3
%
%% Will only show Schwarz, Muench and "the AAS Journals Data Scientist 
%% collaboration" on the front page of this example manuscript.
%%
%% Note that all of the author will be shown in the published article.
%% This feature is meant to be used prior to acceptance to make the
%% front end of a long author article more manageable. Please do not use
%% this functionality for manuscripts with less than 20 authors. Conversely,
%% please do use this when the number of authors exceeds 40.
%%
%% Use \allauthors at the manuscript end to show the full author list.
%% This command should only be used with \AuthorCollaborationLimit is used.

%% The following command can be used to set the latex table counters.  It
%% is needed in this document because it uses a mix of latex tabular and
%% AASTeX deluxetables.  In general it should not be needed.
%\setcounter{table}{1}

%%%%%%%%%%%%%%%%%%%%%%%%%%%%%%%%%%%%%%%%%%%%%%%%%%%%%%%%%%%%%%%%%%%%%%%%%%%%%%%%
%%
%% The following section outlines numerous optional output that
%% can be displayed in the front matter or as running meta-data.
%%
%% If you wish, you may supply running head information, although
%% this information may be modified by the editorial offices.
\shorttitle{NIRCam Lab Stability Studies}
\shortauthors{Schlawin et al.}
%%
%% You can add a light gray and diagonal water-mark to the first page 
%% with this command:
% \watermark{text}
%% where "text", e.g. DRAFT, is the text to appear.  If the text is 
%% long you can control the water-mark size with:
%  \setwatermarkfontsize{dimension}
%% where dimension is any recognized LaTeX dimension, e.g. pt, in, etc.
%%
%%%%%%%%%%%%%%%%%%%%%%%%%%%%%%%%%%%%%%%%%%%%%%%%%%%%%%%%%%%%%%%%%%%%%%%%%%%%%%%%

%% This is the end of the preamble.  Indicate the beginning of the
%% manuscript itself with \begin{document}.

\begin{document}

\title{Lessons Learned for JWST Transiting Planet Time Series from Ground-based Studies of the NIRCam Detectors}

%% LaTeX will automatically break titles if they run longer than
%% one line. However, you may use \\ to force a line break if
%% you desire. In v6.1 you can include a footnote in the title.

%% A significant change from earlier AASTEX versions is in the structure for 
%% calling author and affilations. The change was necessary to implement 
%% autoindexing of affilations which prior was a manual process that could 
%% easily be tedious in large author manuscripts.
%%
%% The \author command is the same as before except it now takes an optional
%% arguement which is the 16 digit ORCID. The syntax is:
%% \author[xxxx-xxxx-xxxx-xxxx]{Author Name}
%%
%% This will hyperlink the author name to the author's ORCID page. Note that
%% during compilation, LaTeX will do some limited checking of the format of
%% the ID to make sure it is valid.
%%
%% Use \affiliation for affiliation information. The old \affil is now aliased
%% to \affiliation. AASTeX v6.1 will automatically index these in the header.
%% When a duplicate is found its index will be the same as its previous entry.
%%
%% Note that \altaffilmark and \altaffiltext have been removed and thus 
%% can not be used to document secondary affiliations. If they are used latex
%% will issue a specific error message and quit. Please use multiple 
%% \affiliation calls for to document more than one affiliation.
%%
%% The new \altaffiliation can be used to indicate some secondary information
%% such as fellowships. This command produces a non-numeric footnote that is
%% set away from the numeric \affiliation footnotes.  NOTE that if an
%% \altaffiliation command is used it must come BEFORE the \affiliation call,
%% right after the \author command, in order to place the footnotes in
%% the proper location.
%%
%% Use \email to set provide email addresses. Each \email will appear on its
%% own line so you can put multiple email address in one \email call. A new
%% \correspondingauthor command is available in V6.1 to identify the
%% corresponding author of the manuscript. It is the author's responsibility
%% to make sure this name is also in the author list.
%%
%% While authors can be grouped inside the same \author and \affiliation
%% commands it is better to have a single author for each. This allows for
%% one to exploit all the new benefits and should make book-keeping easier.
%%
%% If done correctly the peer review system will be able to
%% automatically put the author and affiliation information from the manuscript
%% and save the corresponding author the trouble of entering it by hand.

\correspondingauthor{Everett Schlawin}
\email{eas342 AT EMAIL Dot Arizona .edu}

\author[0000-0001-8291-6490]{Everett Schlawin}
\affiliation{Steward Observatory \\
933 North Cherry Avenue \\
Tucson, AZ 85721, USA}

\author{Rafia Bushra}
\affiliation{Steward Observatory \\
933 North Cherry Avenue \\
Tucson, AZ 85721, USA}

\author{Stephanie Striegel}
\affiliation{Department of Physics \& Astronomy \\
San Jose State University \\
One Washington Square \\
San Jose, CA 95192, USA}
\affiliation{NASA Ames Research Center \\
Space Science and Astrobiology Division \\
Moffett Field, CA 94035, USA}

\author{Xander Levinson}
\affiliation{Department of Astronomy \& Astrophysics \\
University of California, Santa Cruz \\
1156 Highland Street \\
Santa Cruz, CA 95062, USA}
\affiliation{NASA Ames Research Center \\
Space Science and Astrobiology Division \\
Moffett Field, CA 94035, USA}

\author{Jarron Leisenring}
\affiliation{Steward Observatory \\
933 North Cherry Avenue \\
Tucson, AZ 85721, USA}

\author{Karl Misselt}
\affiliation{Steward Observatory \\
933 North Cherry Avenue \\
Tucson, AZ 85721, USA}

\author{Douglas Kelly}
\affiliation{Steward Observatory \\
933 North Cherry Avenue \\
Tucson, AZ 85721, USA}

\author{Thomas Beatty}
\affiliation{Steward Observatory \\
933 North Cherry Avenue \\
Tucson, AZ 85721, USA}



\author{Thomas P Greene}
\affiliation{NASA Ames Research Center \\
Space Science and Astrobiology Division \\
Moffett Field, CA 94035, USA}

\author{Marcia Rieke}
\affiliation{Steward Observatory \\
933 North Cherry Avenue \\
Tucson, AZ 85721, USA}


%% Note that the \and command from previous versions of AASTeX is now
%% depreciated in this version as it is no longer necessary. AASTeX 
%% automatically takes care of all commas and "and"s between authors names.

%% AASTeX 6.1 has the new \collaboration and \nocollaboration commands to
%% provide the collaboration status of a group of authors. These commands 
%% can be used either before or after the list of corresponding authors. The
%% argument for \collaboration is the collaboration identifier. Authors are
%% encouraged to surround collaboration identifiers with ()s. The 
%% \nocollaboration command takes no argument and exists to indicate that
%% the nearby authors are not part of surrounding collaborations.

%% Mark off the abstract in the ``abstract'' environment. 
\begin{abstract}

JWST transmission and emission spectra of transiting exoplanets will provide invaluable glimpses at exoplanet atmospheres.
These spectra will reveal the composition and temperature structure at a level never achieved before.
This promising science from JWST, however, will require exquisite precision and understanding of systematic errors that can impact the time series of planets crossing in front of and behind their host stars.
This is especially true if JWST is used to search for biosignatures in temperature atmospheres on Earth-sized planets that may contain liquid water.
Here, we provide the lessons learned from ground-based characterization of the NIRCam H2RG detectors, which will be used for exoplanet spectra.
%and are the same type of detectors used in JWST's NIRISS and NIRSpec instruments.
We summarize the lessons learned from tests at NASA Goddard, NASA Johnson and the University of Arizona detector labs.

\end{abstract}

%% Keywords should appear after the \end{abstract} command. 
%% See the online documentation for the full list of available subject
%% keywords and the rules for their use.
\keywords{stars: atmospheres --- stars: individual (\objectname{KIC 12557548}) ---
stars: variables: general}

%% From the front matter, we move on to the body of the paper.
%% Sections are demarcated by \section and \subsection, respectively.
%% Observe the use of the LaTeX \label
%% command after the \subsection to give a symbolic KEY to the
%% subsection for cross-referencing in a \ref command.
%% You can use LaTeX's \ref and \label commands to keep track of
%% cross-references to sections, equations, tables, and figures.
%% That way, if you change the order of any elements, LaTeX will
%% automatically renumber them.

%% We recommend that authors also use the natbib \citep
%% and \citet commands to identify citations.  The citations are
%% tied to the reference list via symbolic KEYs. The KEY corresponds
%% to the KEY in the \bibitem in the reference list below. 

\section{Introduction} \label{sec:intro}

JWST will provide powerful new measurements of exoplanet atmospheres with spectroscopy from 0.6 to 12 $\mu$m
\citep{beichman2014pasp,greene2016jwst_trans,howe2017informationJWST,barstow2015jwstSystematics,schlawin2018JWSTforecasts}.
The sensitivity of JWST combined with its unprecedented wavelength coverage on exoplanets will enable it to measure the abundance of carbon-bearing molecules (CO, CO$2$ and CH$_4$) as well as study cooler, smaller planets than previously characterized.

JWST has also been considered for observations of potentially habitable Earth-like planets that orbit their stars at a distance where water could be in liquid form on their surface.
The TRAPPIST-1 system \citep{gillon2016trappist1Discovery,gillon2017trappist-1sevenp} presents an exciting opportunity to study Earth-sized stars orbiting a 0.1~R$_\odot$ M-type star.
By putting together 10 to 30 transits, it will be possible to collect enough photons to detect CO$_2$ \citep{barstow2016trappist1habitable,krissansen-totton2018trappist1eJWST}.
Optimistically, 30 transits may contain enough photons to detect O$_3$ in the atmosphere of TRAPPIST-1 d with JWST if clouds have a minimal impact on its atmosphere \citep{barstow2016trappist1habitable}.
Alternatively, 10 transits of TRAPPIST-1e could be sufficient study if the planet's atmosphere has CO$_2$ and CH$_4$ biosignatures similar to early-Earth \citep{krissansen-totton2018trappist1eJWST}.
A critical question in studying planets like TRAPPIST-1 d/e, however, is whether photon-limited performance is possible with JWST observations.

Experience with the Hubble Space Telescope (HST), \spitzer, and \kepler\ shows that many systematics can affect high precision time series and prevent photon-limited performance unless corrected for \citep[e.g.][]{beichman2014pasp}.
Many of these effects are detector-related including charge trapping on HST's detector \citep{berta2012flat_gj1214,zhou2017chargeTrap}, intra-pixel sensitivity on the short wavelength bands of \spitzer\ IRAC \citep{moralesCalderon2006LdwarfsWeatherIPC} and sensitivity to pointing jitter with \kepler\footnote{https://keplerscience.arc.nasa.gov/K2/Performance.shtml} \citep{beichman2014pasp}.

The NIRCam instrument \citep{rieke2005nircamSPIE} contains a slitless grism mode with its long wavelength channel \citep{greene2017jatisNIRCam}.
This slitless grism mode is analogous to the WFC3 grism on HST that has been successfully employed on many transiting planets \citep[e.g.][]{deming13,kreidberg2014wasp43,sing2016continuum,wakeford2017hatp26}.
The NIRCam grism mode will collect moderate resolution R$\sim$1100-1700 spectra over the wavelength range from 2.4 to 5.0~$\mu$m.
This grism mode can be used simultaneously with short wavelength weak lens imaging or the Dispersed Hartmann Sensor mode \citep{schlawin2017dhs}. If it becomes an approved and implemented mode, the DHS will permit spectroscopy simultaneous from 1.0 to 2.0~$\mu$m because of a dichroic beamsplitter that divides the light from NIRCam into two different channels.
The DHS mode will permit time series spectroscopy on very bright targets ($K_S \sim 1$) inaccessible by other modes.

Section \ref{sec:detectorPrimer} gives a review of NIRCam's detector operation.
Section \ref{sec:knownEffects} lists the known systematic effects that can impeded high precision time series and increase the noise above the photon and read noise.
Section \ref{sec:experiments} describes the experiments on the ground that are used to prepare for JWST observations in flight.


\section{A Primer on Detector Operation}\label{sec:detectorPrimer}

\subsection{Photosensitive Operation}
\textcolor{red}{\textbf{NOTE: This section needs checking for factual accuracy by a detector expert}}

It is helpful to review the basic operation of a NIR detector \citep[e.g.][]{rieke2007irDetectorReview} in order to understand the systematics affecting time series.
Figure \ref{fig:npSchematic} shows a schematic of an NP semiconductor junction for a H2RG detector from \citet{smith2008imgPersistence}.
The crystalline HgCdTe structure has 4 valence electrons that form covalent bonds in a tetrahedral structure.
By varying the relative amount of Hg, Cd and Te (doping), it is possible to increase or decrease the number of electrons to create mobile charge carriers that move along the crystaline lattice.
The N-type semiconductor is a negatively-doped HgCdTe crystalline structure that has negative charge carriers (electrons).
The P-type semiconductor is a positively-doped HgCdTe crystalline structure that has positive charge carriers (vacant holes in the lattice structure than can move like electrons).

\begin{figure}[!hbtp]
\centering
\includegraphics[width=.99\columnwidth]{ideal_photodiode.pdf}
\caption{A cartoon schematic of a H2RG detector inspired by \citet{smith2008imgPersistence} and \citet{tulloch2018persistenceH2RG}.
The orange and blue purple colors depict the N-type and P-type semiconductors, respectively.
The undepleted zones of the semiconductor (darker colors with - and + signs) contain mobile charge carriers whereas the depletion region (lighter color without - and +) has no mobile carriers and all electrons are in the valence state.
The depletion region is widest after detector reset and shrinks as the detector is illuminated with photons that create electron-hole pairs.
The electron-hole pairs migrate across the depletion region towards the undepleted zones due to an electric field.
The voltage can be measured non-destructively as the well fills and then reset before saturating.}\label{fig:npSchematic}
\end{figure}



The junction of these two materials enables electrons to flow from the N-type material to the P-type material where it can complete a valence shell in the semiconductor material.
This produces a negative charge on the P-type side.
Conversely, the positively charged holes from the P-type material flow to the N-type material to complete the valence shell and this produces a net positive charge.
The resulting voltage difference is called the contact potential.
In the region where electrons have completed the valence shells in the P-type material and holes in the N-type material, the valence shells are complete.
This region spanning across the two semiconductor materials is called the depletion region (pictured in Figure \ref{fig:npSchematic}), because it contains no mobile charge carriers.
The positive net charge on the N-type material and negative charge in the P-type material causes a voltage (potential) difference across the detector.
It should be noted that the ``depletion region'' is depleted in {\it mobile charge carriers}, but is {\it not} depleted in net charge because has a net positive charge in the N-type depletion region and negative charge in the P-type depletion region.

The NP junction in Figure \ref{fig:npSchematic} is ``reverse-biased'' with a positive potential on the N-type material and negative potential on the P-type.
The positive potential on the N-type material attracts the negative charge carriers (electrons) whereas the negative potential on the P-type material attracts the  positive charge carriers (holes).
The reverse bias thus reduces the number of mobile charge carriers and increases the size of the depletion region.
The increased depletion region size also increases the well depth or capacity of electrons that the detector may collect before saturation.
The reverse bias is applied at each reset at the beginning of a detector's integration.
Figure \ref{fig:npSchematic} shows the initial reverse voltage that is applied during a pixel reset. 
The physical size of the N-type HgCdTe semiconductor is much larger than the P-type HgCdTe because the density of carriers is lower in the N-type.

Note that Figure \ref{fig:npSchematic} shows the mobile charge carriers only.
Textbook NP junction schematics often show the net charge and thus symbolize the regions with charge carriers as empty and the depletion region with ``+'' symbols on the N-type semiconductor and ``-'' symbols on the P-type˚ \citep[e.g.][]{halliday2004physicsText}.
The schematic in Figure \ref{fig:npSchematic} instead shows the mobile charge carriers as ``+'' and ``-'' because it is easier to visualize the charge trapping effect and motion of mobile charges this way \citep[e.g.][]{smith2008imgPersistence}.

When an incoming photon strikes the N-type depletion region, it will excite an electron from the valence to conduction band which creates an electron-hole pair \citep{rieke2007irDetectorReview}.
The depletion region has a net electric field pointed toward the P-type side, which controls the direction electrons and holes will migrate.
The electron moves to the extremity of the depletion zone in the N-type material whereas the positive hole generated crosses the NP junction and adds a mobile charge to the P-type side.
The incoming photon has the effect of decreasing the size of the depletion region.
A second photon will excite another electron-hole pair so that a hole moves to the P-type side.
This continues and decreases the size of the depletion region.
The diffusion of charge carriers across the junction changes the voltage across the NP junction, which can be measured with a sensitive amplifier circuit.

As the size of the depletion region decreases, the detector becomes decreasingly sensitive to new photons.
Thus, the NIRCam HgCdTe detectors are never strictly linear.
Eventually, with enough photons, the depletion region is too small to create electron-hole pairs with incoming photons.
This is where the detector approaches saturation and no longer changes voltage with more light.

\subsection{Readout Circuit}\label{sec:readout}

The accumulated charge from absorbed photons in the detector is measured with a multiplexor (MUX) that form a readout integrated circuit (ROIC) \citep{rieke2007irDetectorReview}.
This readout circuitry allows the signal to be measured non-destructively up the ramp in so-called ``multiaccum" mode.

A sensitive amplifier converts the measured voltage potential (as compared with the bias level after reset) into a signal.
The electronic circuit that digitizes the voltages is tuned to ensure the 16 bits (65,536 possible Data Numbers) measure the voltage across the NP junction from the value just after reset (before photons are detected) to the saturation voltage where the maximum number of photons are collected and no more electron-hole pairs can be produced in the depletion region.
The minimum voltage is several thousands of counts above zero to ensure that fluctuations in the bias signal do not go below 0.
The 65,536 DN level is tuned to be just above the HgCdTe full well capacity to measure full saturation of a pixel.




\section{Known Detector Effects}\label{sec:knownEffects}
Section \ref{sec:detectorPrimer} describes the operation of a HgCdTe array.
However, there are many non-ideal effects that can affect the stability of a light curve of a transiting exoplanet.
Here, we review many of the known noise sources that could impact the stability of a time series beyond photon counting statistics. These known detector effects that can impact stability are:

\begin{itemize}[noitemsep]
	\item \textbf{Intrapixel sensitivity:} - including sub-pixel flat fielding issues and sub-pixel crosshatching on the detectors \citep{shapiro2018crosshatch}
	\item \textbf{Pre-amp reset offsets:} (an Asic-related effect) 
The resets cause discontinuities in the pedestal level of all pixels between frames. This is an effect where averaging over more pixels does not decrease the noise, but reference pixel or background subtraction can. These are discussed in Section \ref{sec:preAmp}.
	\item \textbf{1/f noise:} This is a noise source where most of the noise power is concentrated as low frequencies. 1/f noise causes spatial correlations in the fast-read direction (along detector rows). It is highly correlated between amplifiers, suggesting that it is caused by a common reference voltage. 1/f noise can be reduced using reference pixels
	\item \textbf{Even/odd offsets:} Alternating columns have different bias levels. Additionally, there are offsets between columns that can change from frame to frame, especially on NIRCam's LW detector
	\item \textbf{kTC noise:} A random process due to the unknown amount of charge stored at the reset voltage level
	\item \textbf{Detector Temperature Fluctuations:} The JWST detectors are sensitive to temperature changes on the detectors. For example, laboratory tests show that 100 mK temperature fluctuations can result in $\sim$ 80 ADU changes on the detector that are not corrected by reference pixels \citep{hall2005jwstArrays}. Fortunately, the detectors are actively thermally controlled to keep the temperature fluctuations to mostly $\lesssim$ 1 mK.
The one exception is that the Long Wavelength A-side detector (A5) exhibited $\sim$ 20 min oscillations with an amplitude of 15-20 mK when using the Temperature Monitor Control (TMC) 1 active thermal control.
Fortunately, the TMC 2 control does not undergo these oscillations and will be employed in orbit.
If necessary, the effect of temperature fluctuations can be calibrated to $\pm 1 e^-$ for excursions less than $\pm$ 50 mK \citep{hall2005jwstArrays}.
	\item \textbf{ASIC Temperature Fluctuations:} The asic temperature will change as the operating mode. For example, in the CV3 stability test, the temperature of the A asic changed by $\sim$80 mK when the readout mode was switched from full frame to subarray. This temperature change is expected to be smaller in flight due to adjustments in ?? (Marcia said in 2018-11-02 meeting)
	\item \textbf{Elevated Columns} There is correlated noise along a column in many frames. These elevated columns can sometimes even move with time.
	\item \textbf{Random Telegraph Noise (RTN)} Some of the pixels will exhibit spontaneous changes jumps in signal even with no illumination. Fortunately, RTN appears in only $\sim$1000 pixels out of 4$\times 10^6$ on the array.
	\item \textbf{Detector Fringing} \textbf{\textcolor{red}{Add more details and citations}} The thin $\sim 5\mu$m layers of the H2RG detectors will cause interference patterns on the array due to constructive and destructive interference of light at wavelengths similar to the thickness of the layers. If the detector fringe pattern changes with time, it will change the throughput of a grism image.
	\item \textbf{Amplifier Boundary Discontinuities} The NIRCam detectors can be read out in full frame, stripe and window modes. The full frame and stripe modes make use of 4 amplifiers that can record or reset the accumulated electrons in 4 pixels simultaneously in parallel. The parallel operation of the amplifiers reduces the frame time and reset time by a factor of 4 but can introduce subtle voltage biases between regions of the detector. If these voltage biases are not corrected, they can potentially introduce discontinuities in a spectrum, which crosses amplifier boundaries.
\end{itemize}

It should be noted that the detector effects related to the asic circuit can be specific to the ``personality'' of each asic. The asics all have different asic load files which are tuned to minimize the power consumption and dark current in a particular asic. Since the asic load files are specific to each detector, they may exhibit different noise properties.
Some caution should be taken when comparing noise tests with engineering grade or rejected detectors with the flight detectors.

\subsection{Pre-amp Reset Offsets from the Lockheed CV Test}\label{sec:preAmp}

The NIRCam detectors' pre-amplifiers (pre-amps) have a reference voltage that can drift with time.
This can affect the pedestal level of all pixels over long ($\gtrsim$ 20 minute timescales).
These pre-amplifier levels can either be reset at the beginning of an integration with samples up the ramp or with each frame up the ramp.
The NIRCam detectors are usually set to reset once per frame to prevent large amplifier drifts over an integration \citep{robberto2014refPixPreAmp}.
However, one set of darks was collected at Lockheed CV with a reset only once per integration to study the effects of the pre-amplifier resets.

In the Lockheed CV test, the pedestal level of all pixels could be seen to drift over 19.3 minute dark integrations by 5 to 100 DN.
The drift was usually linear over these 19 minute exposures, but in one case was observed to drift downward from 100 DN to -50 DN and back to 100 DN \citep{robberto2014refPixPreAmp}.
The timescales for pre-amp resets are relevant to exoplanet time series, which have transit durations of $\sim$0.5 to 3 hours for typical targets.
However, reference pixel subtraction can efficiently remove pre-amp drifts, so with proper data reduction and subarrays that include reference pixels, the pre-amp drifts can be dramatically reduced.
It is also likely that background subtraction can mitigate the pre-amp reset offsets from time series integrations.
The NIRCam grism subarrays and dispersed source location are placed at the edge of the NIRCam Long Wavelength detector to ensure reference pixels are saved with active pixels.

Figure \ref{fig:ampResetDark} shows an example time series of the reference pixels for a dark exposure.
The pedestal/bias level of the reference pixels undergoes a sharp jump between each frame (10.7 seconds in duration).
These pre-amp resets per frame causes discontinuities over every frame within an integration, but there is an upside:
Over long timescales (10$^3$ seconds), these resets pre-amp resets will keep the bias drifts hovering around zero without any long slopes or drifts.

\begin{figure}[!hbtp]
\centering
\includegraphics[width=.49\columnwidth]{allamps.pdf}
\includegraphics[width=.49\columnwidth]{allamps_long_dark.pdf}
\caption{Reference pixel time series for a two frame integration (Left) and a long dark integration of 108 frames (Right). Each of the plots show that the pre-amplifier resets between frames in an integration cause discontinuities in the reference signal. Amplifiers 1 and 4 track the time all through a frame with the side reference pixels, whereas amplifiers 2 and 3 in the middle only contain information on the bottom and top reference pixels at the beginning and ending of a frame.
The long dark exposure example of 108 integrations shows the behavior over long timescales.}\label{fig:ampResetDark}
\end{figure}

\subsection{Charge Trapping}
All HgCdTe detectors show a signal after bright illumination even if the illumination is removed; this is called persistence.
The physical mechanism that explain persistence is that charges are trapped in the depletion region after illumination by photons as the detector well fills \citep{smith2008imgPersistence,leisenring2016persistence}.
After the reverse-bias voltage is applied (a reset) in an ideal detector, the depletion region will be devoid of mobile charge carriers (electrons and holes).
However, in real detectors, some charges are trapped within the depletion region.
During a future integration, these charges will be released and shrink the size of the depletion region, filling a pixel with spurious signal (Data Numbers) not related to that integration's illumination by photons.

Figure \ref{fig:npSchematicTraps} shows a schematic of a detector that has charge traps in the substrate.
After illumination by a source, these traps will fill with charge that does not migrate across the depletion layer.
When the reset voltage is applied to the detector, the trapped charge remains within the depletion layer.
In the second integration in the schematic, charge is released, which reduces the size of the depletion layer faster than the incoming photons would otherwise.
The consequence of charge traps is that there is a deficiency in measured charge on the first integration and an excess of measured charge on the second integration.
Therefore, a constant astrophysical signal can be measured as a time-varying one due to the charge trapping effect.
Over timescales longer than the charge release time, the detector will reach a steady state.

The flight NIRCam detector shows low levels of persistence compared to previous generation detectors.

\begin{figure}[!hbtp]
\centering
\includegraphics[width=.99\columnwidth]{charge_traps_photodiode.pdf}
\caption{A cartoon schematic of a H2RG detector with charge traps inspired by \citet{smith2008imgPersistence}, \citet{tulloch2018persistenceH2RG} and \citet{leisenring2016persistence}.
Charged traps for negative and positive carriers (depicted as round circles) will capture charge before it can flow to the undepleted regions of the detector.
The charge traps do not immediately empty with a detector reset, but instead release at later times causing a spurious signal.
This spurious signal (the shrinking of the depletion region as highlighted with red circles), causes a voltage change in future integrations that appears the same as a true signal from incoming photons.}\label{fig:npSchematicTraps}
\end{figure}

\subsection{Subpixel Crosshatching}

\subsubsection{Subpixel Crosshatching Characterization}
The flat field of the ALONG detector, which is used for NIRCam grism series, has a pronounced crosshatch pattern visible in Figure \ref{fig:crossHatchA5}.
The patterns are located at 23.1$\degree$ , 90.9$\degree$, and 158.6 $\degree$ counter-clockwise from the $+$X direction of the detector.
The angle between these lines are 67.8$\degree$, 67.7$\degree$ and 44.5$\degree$.
These are best visualized with a 2D power spectrum as shown in Figure \ref{fig:crossHatchA5}.

The angles of the crosshatch patterns are likely set by the crystal pattern of HgCdTe.
HgCdTe has a zincblende structure with tetrahedral bond angles where each Hg or Cd atom is surrounded by 4 Te atoms \citep{gemain2012mercVacanciesHgCdTe}.
One possible projection of a zincblende structure is shown in Figure \ref{fig:crossHatchA5}.
In this perspective, the projected bond angles form angles fo 67.8 $\degree$, 67.8 $\degree$ and 44.4$\degree$, close to the observed crosshatch angles.
We therefore deduce that the crosshatch pattern is mostly likely related to the crystal lattice of the HgCdTe substrate and not the pixels or circuitry.

The crosshatch structure of the detectors extends down to the subpixel level, so it will not be fully corrected with a flat field division.
This sub-pixel structure has been imaged with microscopy on a Euclid HgCdTe detector \citep{shapiro2018crosshatch}.
The width of the structures can be estimated from the behavior for crosshatch lines as they cross pixel boundaries \citep{ninan2019crosshatchHPF}.
We estimate that the crosshatch pattern oriented at 90.9$\degree$ crosses 1 horizontal pixel for every 64 vertical pixels.
While crossing the boundary, there are $\approx 38$ rows where the crosshatch pattern spans more than 2 pixels.
If the crosshatch pattern has a tophat response function, then these 38 rows where the pattern spans two pixels imply a tophat full width of 0.6 pixels or a physical width of 10.8~$\mu$m for an 18~$\mu$m pixel pitch.
This is more than twice the estimate from \citet{ninan2019crosshatchHPF} for an HgCdTe used on the Habitable Planet Finder.
It is possible that the crosshatch pattern's width varies among detectors or that a tophat function is a poor approximation of the actual subpixel response.
Within NIRCam detectors, there are large variations in the strength and orientation of crosshatch features.




\begin{figure}[!hbtp]
\centering
\includegraphics[width=.49\columnwidth]{crosshatch_zoom.pdf}
\includegraphics[width=.49\columnwidth]{crosshatch_2d_power.pdf}
\includegraphics[width=.3\columnwidth]{tetrahedral_lattice_projection.png}
\caption{
The A5 detector's flat field has a pronounced crosshatch pattern at 23.1$\degree$ , 90.9$\degree$, and 158.6 $\degree$ CCW from the positive X direction.
The crosshatch pattern is visible in a zoom-in of the pixel flat field (top left).
The 2D power spectrum of the flat field (top right) shows a continuum of frequencies aligned with the three crosshatch angles.
Note the lines in frequency plane are perpendicular to the crosshatch directions in the image plane.
These three relative angles (67.8$\degree$, 67.7$\degree$ and 44.5$\degree$) are similar to one projection of a tetrahedral (zincblende) lattice structure of HgCdTe (bottom, 67.8$\degree$, 67.8$\degree$, 44.4$\degree$).
}\label{fig:crossHatchA5}
\end{figure}

The crosshatch pattern is wavelength dependent, as seen in Figure \ref{fig:crossHatchWavelengthDep}, where the throughput variations are the largest for short wavelengths (resolving crystal structures) and smallest for the long wavelengths.
We also fit the crosshatch pattern.

\begin{figure}[!hbtp]
\centering
\includegraphics[width=.49\columnwidth]{cross_sec_of_crosshatch.pdf}
\includegraphics[width=.49\columnwidth]{fourier_power_from_fit.pdf}
\caption{
The crosshatch pattern changes as a function of wavelength, possibly because the shorter wavelengths resolve the structure better.
The shorter wavelengths have deeper crosshatch troughs (left) and higher Fourier amplitudes (right).
}\label{fig:crossHatchWavelengthDep}
\end{figure}


\subsubsection{Crosshatch Modeling}
We model the crosshatch pattern in the Fourier space because it has a smoother pattern as a function of spatial frequency than in spatial coordinates as seen in Figure \ref{fig:crossHatchA5}.
We divide the 2D Fourier power spectral density by the number of pixels in an input image.
This was experimentally determined to give a power spectral density that does not change with the dimensions of the input image.
\textcolor{red}{Can we justify this more?}

We model the two dimensional Fourier power spectral density $f(k_{x,i}',k_{y,i}')$ for one angle $\theta_i$ as
\begin{equation}\label{eq:analyticPSDradial}
f(k_{x,i}',k_{y,i}', a_i, b,c) = a_i \exp{\left(- |k_{x,i}'| / b \right)} \left( \frac{1}{(0.5 c)^2 + k_y'^2} \right),
\end{equation}
where $k_x'$ and $k_y'$ are rotated frequency coordinates in the parallel and perpendicular directions, $a$ is the amplitude, $b$ is the parallel exponential constant and $c$ is the Lorentzian full width at half maximum of the perpendicular dependence.
For each of the three angles, there is a set of rotated coordinates centered at $(k_{x,0},k_{y,0})$= (0,0) following:
\begin{equation}
k_x'(\theta_i) = k_x \cos{\theta_i} + k_y \sin{\theta_i}
\end{equation}
and
\begin{equation}
k_y'(\theta_i) = -k_x  \sin{\theta_i} + k_y \cos{\theta_i}.
\end{equation}
We include three angles so that the total power spectral density is
\begin{equation}
f(k_x,k_y) = \sum_{i=1}^{i=3} f_i(k_{x,i}'(\theta_i),k_{y,i}'(\theta_i),a_i,b,c).
\end{equation}

This model has 8 free parameters: three angles ($\theta_i$, three amplitudes ($a_i$), a joint parallel exponential constant ($b$) and joint Lorentzian width ($c$).
We only fit the region of the power spectral density above frequencies of 0.05 px$^{-1}$ to focus on the high frequency component of the flat field where the crosshatch is most prevalent.
For the F300M filter and ALONG detector, we find 
for $\theta_1 = 0.9013 \pm 0.0004 \degree, \theta_2 = 68.6041 \pm 0.0007 \degree, \theta_3 = 113.0741 \pm 0.0007, a_1 = 3.93 \pm 0.01 \times 10^{-7}, a_2 = 2.305 \pm 0.007 \times 10^{-7}, a_3 = 2.540 \pm 0.007 \times 10^{-7}, b=0.3243 \pm 0.0007 $ px$^{-1}$ and $c=2.44 \pm 0.03 \times 10^{-2}$ px$^{-1}$.
The uncertainties were simply derived from the diagonals of the covariance matrix and likely are underestimated.

\subsubsection{Subpixel Crosshatching Simulation}\label{sec:CrosshatchSim}

The subpixel crosshatch pattern can potentially introduce time-variable noise as the point spread function drifts with telescope pointing drifts, long timescale jitter.
NIRCam observations are expected to have a root-mean-square deviation of 6.0 mas in each axis when measured in 15 second intervals over a 10,000 second observation.\footnote{See \url{https://jwst-docs.stsci.edu/display/JTI/JWST+Pointing+Performance\#JWSTPointingPerformance-Pointing\_stabilityPointingstability}}

We simulate the effect of this jitter on NIRCam time series observations with the following steps:
\begin{itemize}
	\item Fit the existing flat field to a crosshatch model in the Fourier domain
	\item Extrapolate the crosshatch model to high frequencies to estimate the sub-pixel structure
	\item Simulate a PSF using \texttt{webbpsf} \citep{perrin2014webbpsf}
	\item Multiply the PSF by the simulated flat field
	\item Scan the PSF along the X and Y directions up to a magnitude of 6 mas
	\item Bin the simulated images into the native pixel size
	\item Divide by the flat field
	\item Measure the relative flux with a photometric aperture
\end{itemize}



We take the exponential model with best-fit parameters for the F300M filter and extend the exponential function out to an oversampling factor of 60 times the LW pixels.
We evaluate equation \ref{eq:analyticPSDradial} for frequencies in the oversampled image from 0 px$^{-1}$ to 30 px$^{-1}$ and multiply this by the number of pixels and the square of the oversampling factor (ie. $60^2$) to convert from the scale-invariant Power Spectral Density to the simulated PSD.
We then assign random phases uniformly from 0 to 2$\pi$ for the complex Fourier plane because the original image's complex phase distribution is similar to uniform.
Finally, we take the real part of the inverse Fourier transform to create an oversampled image, as shown in \ref{fig:crossHatchSimPSF}.
For comparison to the original flat in Figure \ref{fig:crossHatchA5}, we binned to the oversampled flat field to the native LW pixel resolution of $\sim$0.063 \arcsec.
This simulated pixel flat field with dimensions of 26 by 26 LW pixels has a peak of 1.15 to trough of 0.90 with a robust standard deviation of 0.05.
For comparison, the middle of the original F300M flat for the ALONG detector has a robust standard deviation 0.06.

\begin{figure}[!hbtp]
\centering
\includegraphics[width=.32\columnwidth]{sim_flat_Oversampled_plot.pdf}
\includegraphics[width=.32\columnwidth]{sim_flat_Pixel_plot.pdf}
\includegraphics[width=.32\columnwidth]{psf_with_phot_ap_with_crosshatch.pdf}
\caption{
The steps of the sub-pixel crosshatch photometry simulation are shown here.
We simulate an over-sampled flat (left) by extrapolating the power spectrum to frequencies above the pixel sampling level.
This simulated flat is binned to the native pixel resolution to produce a pixel flat field (middle) that would be used by a standard pipeline that has no sub-pixel corrections.
A \texttt{WebbPSF} point spread function is created at the oversampled resolution, multiplied by the oversampled flat field and then binned to native resolution (right).
Finally, the simulated image is divided by the pixel flat field as would be performed in a standard pipeline reduction.
The simulated PSF shows the source and background apertures used in the simulation, which are centered on the PSF.
}\label{fig:crossHatchSimPSF}
\end{figure}

We next calculate a point spread function using \texttt{webbpsf} \citep{perrin2014webbpsf}, oversampled by a factor of 60.
\texttt{webbpsf} includes an optional blurring due to high frequency pointing jitter.
We set this Gaussian blurring parameter to 1 mas to simulate a worst-case scenario where the pointing drift is dominated by long-timescale behavior with minimal high frequency jitter.
We multiply the oversampled \texttt{webbpsf} source by the simulated oversampled flat and bin this to native pixel resolution to create a simulated observation as shown in Figure \ref{fig:crossHatchSimPSF}.
This simulated observation is divided by the binned simulated flat as would be done by a standard pipeline's flat field correction.

We repeat the last steps of the observation simulation by shifting the over-sampled PSF by sub-pixel amounts.
The subpixel shifts of the over-sampled PSF are performed with \texttt{scipy.ndimage.shift}.
For each shifted PSF, we multiply this by the subpixel crosshatch pattern and bin the result.
These simulated operations represent pointing drift on the subpixel scale.

We calculate aperture photometry with a circular aperture with a radius of 7 pixels and a background annulus from 7 to 10 pixels on each of the observations.
The aperture is re-centered by the same position as the shift direction.
For each sub-pixel pointing drift, the aperture sum is subtracted by the background flux average per pixel multiplied by the pixel area of the source aperture.
While the simulations here include no background flux, we use the background subtraction to simulate the operations applied by a photometry pipeline.
The differential flux between the centered PSF and the shifted one is shown in Figure \ref{fig:subpixScanSimulation}.


\begin{figure}[!hbtp]
\centering
\includegraphics[width=.49\columnwidth]{scan_F300M_x_short_scan.pdf}
\includegraphics[width=.49\columnwidth]{scan_F300M_y_short_scan.pdf}
\caption{
The simulated subpixel crosshatch pattern from Figure \ref{fig:crossHatchSimPSF} for the F300M filter will cause spurious flux changes with small pointing drifts.
Here we show scans in the X direction and Y direction from the center of the image and a subpixel scale.
The amplitude of these scans is 6.3 mas for the $\sim$63 mas/px plate scale of the LW detector.
These are correctable with smooth functions such as polynomials.
}\label{fig:subpixScanSimulation}
\end{figure}

The subpixel crosshatch structure does indeed create flux variations with image motion as shown in Figure \ref{fig:subpixScanSimulation}.
The amplitude of the flux changes is potentially up to 300 ppm.
This is significant when compared to atmospheric features of giant planets ($\lesssim$ 100 ppm) or the transit depth of an Earth-like planet transiting a sun-like star ($\lesssim$ 80 ppm).

Fortunately, the subpixel crosshatch systematic is a smooth function of pointing drift, shown in Figure \ref{fig:subpixScanSimulation}
Smooth functions, such as polynomials, with centroid position will likely 
Centroiding is possible in imaging and spectroscopic modes using PSF fitting or cross-correlation.
When the LW grism time series mode is enabled, SW imaging data will automatically be collected simultaneously enabling the SW centroids to be used to track the motion of the dispersed grism image.
Furthermore, the pixels for time series modes can be characterized in detail because the will be re-used for all time series observations in a given mode.
Target acquisition is expected to achieve centroiding accuracy $\lesssim$ 10 mas ($\lesssim 6$ mas) for unsaturated target acquisition.\footnote{https://jwst-docs.stsci.edu/display/JTI/NIRCam+Time-Series+Imaging+Target+Acquisition}
This ensures that time series observations will reliably return to the same location within the same set of pixels with every visit.

\subsubsection{Position of SubGrism Array}

The crosshatch pattern varies in amplitude and angular dependence from one physical location on the detector to another.
The ALONG detector has more pronounced crosshatching toward the middle of the detector that falls off toward the perimeter.
On the other hand, the BLONG detector has more pronounced crosshatching towards the boundary.
We fit the crosshatch pattern to three regions of the detector:
\begin{itemize}
	\item The 2048$\times$64 SUBGRISM64 nominal position at the bottom of the array. Bottom Left corner is (4,5)
	\item A 2048$\times$64 theoretical subarray at the top of the ALONG detector. Bottom Left corner is (0,1983)
	\item A 2048$\times$64 cutout of the full frame detector that is centered on the grism time series field point.
\end{itemize}

The three subarrays considered are shown in Figure \ref{fig:subpixScanSimulation}.
The position at the top of the ALONG detector has the smallest amplitude of crosshatch pattern in all three directions.
We exclude reference pixels (which form a 4 pixel wide boundary around the detector) and an additional row on the bottom and top of the array because these illuminated rows are anomalous.
The best-fit amplitudes are used as input to the jitter simulations in Section \ref{sec:CrosshatchSim}.
We find that the maximum flux change for a 0.1 pixel shift is 118 ppm for the nominal grism position, 108 ppm for the top of the array and 153 ppm at the full frame grism position.

\begin{figure}[!hbtp]
\centering
\includegraphics[width=0.99\columnwidth]{subg_pos_topGrism64.pdf}\\
\includegraphics[width=0.99\columnwidth]{subg_pos_fullfGrismRegion.pdf}\\
\includegraphics[width=0.99\columnwidth]{subg_pos_subgrism64.pdf}\\
\vspace{0.2in}
\includegraphics[width=0.99\columnwidth]{subg_pos_topGrism64_psd.pdf}\\
\includegraphics[width=0.99\columnwidth]{subg_pos_fullfGrismRegion_psd.pdf}\\
\includegraphics[width=0.99\columnwidth]{subg_pos_subgrism64_psd.pdf}\\
\caption{
The top of the array has the smallest level of crosshatching and the middle has the largest.
This is visible in the flat field images (top three panels) and their power spectral density functions (bottom three panels) for three different sections of the A5 detector: a theoretical ``aperture'' position at the top of the array (first plot from top), the full frame position near Y=512 (second plot from top) and the nominal SUBGRISM64 at the bottom of the array (third plot from top).
The bottom left corner for each subarray is listed in parentheses in the title of the flat field plot and the best-fit amplitudes are listed in the title of the power spectral density plot.
 }\label{fig:subpixScanSimulation}
\end{figure}


\subsection{Amplifier Boundary Discontinuities}

NIRCam's STRIPE mode and full frame mode both employ 4 amplifiers to simultaneously read out or reset 4 pixels at once.
This simultaneous use of 4 amplifiers reduces the frame time by a factor of 4 and thus increases the brightness of objects that can be detected before saturation as well as increase the number of reads up a detector pixel to average down readout noise.
On the flip side, the four amplifiers will have different behaviors such as offsets in the bias level.
The differential biases between the amplifiers will show up as step function changes in the slope of the image at the amplifier boundaries (512, 1024 and 1536), as visible in Figure \ref{fig:ampOffsetsOtisGrismSlope} (top panel).
If a spectrum of a source is extracted without any vertical background extraction, the amplifier offsets cause sharp discontinuities in the summed spectrum.

Fortunately, the amplifier offsets are efficiently removed by reference pixels.
For example, we subtract the median of the 4 reference pixels from each column and this efficiently removes the bias signal from each amplifier shown in Figure \ref{fig:ampOffsetsOtisGrismSlope} (middle panel and bottom panel).
It should be noted, however, that there are gradients in the bias signal within each amplifier. If we simply subtract the median value of all reference pixels within an amplifier without applying them on a per-column basis, there are still discontinuities in the spectrum.
In addition to reference pixels, background subtraction along each column with removes the amplifier boundary effect.
If the boundaries somehow proved to be an issue for time series spectra, the WINDOW subarray mode, which uses a single amplifier (at the cost of 4 times the frame time) is an option.

\begin{figure}[!hbtp]
\centering
\includegraphics[width=.99\columnwidth]{amplifier_offsets_in_otis_lw_grism.pdf}
\caption{An example slope image from the LW OTIS Stability test.
{\it First Panel:} No reference pixel correction is applied, leaving offsets at the amplifier boundaries.
{\it Second Panel:} Averaging all the reference pixels within an amplifier and subtracting the result reduces the offsets between amplifiers but there are still noticeable boundaries at X= 512, 1024 and 1536.
{\it Third Panel:} Reference pixel correction using the 4 non-illuminated pixels in each column significantly reduces offsets between amplifiers, but introduces read noise to each column.
{\it Fourth Panel:} Background subtraction with the mean value of the pixels below Y=15 and above Y=190 eliminates offsets between amplifiers without introducing any substantial new noise source.
\textcolor{red}{Things to try: 1) keep looking for a fringing in the spectrum}
}\label{fig:ampOffsetsOtisGrismSlope}
\end{figure}


\begin{figure}[!hbtp]
\centering
\includegraphics[width=.49\columnwidth]{otis_spec_and_norm.pdf}
\includegraphics[width=.49\columnwidth]{otis_spec_periodogram.pdf}
\caption{The integrated spectrum from Figure \ref{fig:ampOffsetsOtisGrismSlope} shows high frequency noise much far in excess of the spectrum's photon and read noise.
The spectrum is normalized by dividing by a spline fit (bottom left plot) to search for fringing effects
The sca
A periodogram shows a peak near 19 pixels period, however, it is not an overwhelming periodic signature.
\textcolor{red}{double check the scatter in spectrum to photon + read noise}
}\label{fig:integratedOtisGrismSpec}
\end{figure}

\section{Ground-based Experiments}\label{sec:experiments}

Table \ref{tab:testSummary} shows a summary of the ground-based tests that were performed on NIRCam detectors.
The flight NIRCam detectors, controllers, electronics, optics and instrument hardware were used in the Cryogenic Vacuum 3 (CV3) Test 3 as well as OTIS (Optical Telescope Element and Integrated Science).
There were are dewars at the Steward Observatory at the University of Arizona used for time series stability tests.
The GL dewar experiments used flight-like detectors, but a different controller (Leach).
The Asic dewar located at Steward Observatory in Tucson, Arizona used a spare NIRCam detector, a spare asic controller and the flight software, but no flight optics.
These two different kinds of detector electronics can be compared against each other to help isolate detector-related systematics from readout electronics-related systematics.

The instrument modes tested included mostly imaging with narrowband LEDs, but there was also a test with spectroscopy on a broadband source.
The one test with a broadband source was an OTIS stability test at NASA Johnson was a grism spectroscopy mode, whih illuminate the long wavelength detector with a slitless spectrum covering 2.4 to 4.0~$\mu$m for the F322W2 filter, as seen in Figure \ref{fig:ampOffsetsOtisGrismSlope}.
The WLP8 imaging tests are where a point source is de-focused by 8 waves with a weak lens (WLP8), as shown in Figure \ref{fig:WLP8PSF}.
This mode was designed for wavefront sensing, but is also useful for high-precision time series because it spreads the light over many pixels.
The weak lens produces a point spread function (PSF) that is hexagonal like JWST's mirrors and has a flat-to-flat width of $\sim$108 pixels and a vertex-to-vertex width of $\sim$136 pixels.
The hexagonal PSF also has a central vertical peak which can be used for aperture centering.
The tests with dewars did not have NIRCam optics and instead were flood illuminated with a light source that had large spatial variations due to a non-uniform source.
 
\begin{deluxetable*}{ccCrlcc}[b!]
\tablecaption{Summary of Ground-Based Tests}\label{tab:testSummary}
\tabletypesize{\footnotesize}
%\tablecolumns{7}
%\tablenum{2}
\tablewidth{0pt}
\tablehead{
\colhead{Name} &
\colhead{Location} &
\colhead{Test Year} &
\colhead{Best Stability} & 
\colhead{Light Source} & 
\colhead{Instrument Mode} &
\colhead{Flight Hardware?} \\
\colhead{} &
\colhead{} &
\colhead{(YYYY)} &
\colhead{(ppm)} &
\colhead{} &
\colhead{} &
\colhead{}
}
\startdata
Lockheed CV& Lockheed Martin& 2014??    & \nodata   & Darks         & Long Darks & \nodata \\ 
Reset per int test & Lockheed Martin& 2014??    & \nodata   & Darks         & Long Darks & Y \\ 
CV 3       & NASA Goddard  & 2016       & 500   	& LED           & WLP8 imaging & Y\\
OTIS       & NASA Johnson  & 2017       &   2200    	& Continuum     & Grism spectroscopy & Y\\
OTIS       & NASA Johnson  & 2017       &   21,000    	& Continuum     & WLP8 Imaging & Y \\
GL dewar   & Steward Obs.  & 2015-2016  &       	& LED	        & flood-illuminated imaging & N \\
Asic dewar & Steward Obs.  & 2017-2019  &       	& LED           & pinhole flood-illuminated imaging & N\\
\enddata
\tablecomments{Cryogenic vacuum (CV), Optical Telescope element and Integrated Science (OTIS).}
\end{deluxetable*}

\clearpage
\subsection{Long Darks}\label{sec:longDarks}

It is instructive to look at dark frames to isolate the contributions of the read-out electronics, including the side-car asic, and how much they contribute to the error budget.
The dark frames do not have issues with the photometric stability of illumination lamps, which is a persistent challenge with ground-based tests.
They also do not contain photon noise in the background, which can mask the electronic noise sources.
We study here the contributions from read noise, 1/f noise and dark current.
This experiment with dark frames does not test subtle changes in behavior of the electronic read noise under illumination, such as cross-talk where pixels that have no direct flux from a source in an image correlate with other regions of the detector with high flux rates.

A standard long dark exposure used for detector characterization is a single integration that is 108 groups of 1 frame each (so-called RAPID mode).
The detectors are read in full frame mode so that each frame is 10.7 sec in duration with 4 output amplifiers in use, thus the total time for the exposure is 19.3 min.
We use one of these long dark integrations to create simulated time series as if it were observing many consecutive integrations in full frame mode.
The long darks are broken into 54 sequential pairs of images that are subtracted from each other (integration 1 minus integration 0, 3 minus 2, 5 minus 4 etc. in a zero-based counting scheme).
Figure \ref{fig:longDarkPhot} shows an example dark read pair from frame 55 minus frame 54.
In the raw data, there are 4 prominent vertical stripes that are 512 pixels wide and 2048 pixels tall due to offsets in the read-out amplifiers.
These amplifier offsets are caused by drifts in the reference voltage that are reset once per frame and vary from amplifier to amplifier.
Additionally, there are obvious correlations along the fast-read (X) direction due to 1/f noise that appear as horizontal stripes in the image.

The ramps-to-slope pipeline \texttt{ncdhas} or the STScI JWST pipeline will apply reference pixel correction, bad pixels masks and non-linearity correction before fitting the groups up the ramp to a line.
In this work, we use \texttt{ncdhas}.
As seen in Figure \ref{fig:longDarkPhot}, reference pixels can be used to reduce the large offsets between amplifiers.
The reference pixels (making up a 4 pixel boundary on the edges of the detector) are un-illuminated and can therefore be used to measure common-mode electronic noise without being affected by photon counting noise.
The reference pixels are most useful for subtracting the offsets between amplifiers because there are 8 rows of 512 pixels or a total of 4096 reference pixels to average for each amplifier at the bottom and top of the arrays.
There are an additional 4 columns on the left and right side boundaries (8176 pixels) that can be used to further correct the left and right amplifier offests.
However, the 1/f noise that correlates along the fast-read (X) direction is more difficult to remove with reference pixels.
Only 4 pixels on the left and 4 pixels on the right side may be used to subtract the 1/f noise in a given row.
Therefore, significantly residuals remain along the fast-read (X) direction in the reference-corrected read pair shown in Figure \ref{fig:longDarkPhot}.
These side reference pixels also cannot give any information about the read noise correlations that happen on shorter length scales than 512 pixels (5 ms).

\begin{figure*}[!hbtp]
\centering
\includegraphics[width=.32\columnwidth]{ap_labels_dark_NRCALONG-DARK-7235074213_1.pdf}
\includegraphics[width=.32\columnwidth]{ap_labels_darkRed_NRCALONG-DARK-7235074213_1.pdf}
\includegraphics[width=.32\columnwidth]{ap_labels_row_by_row_group_sub_aps_1_NRCALONG-DARK-7235074213_1.pdf}
\includegraphics[width=.32\columnwidth]{long_dark_dark.pdf}
\includegraphics[width=.32\columnwidth]{long_dark_darkRed.pdf}
\includegraphics[width=.32\columnwidth]{long_dark_darkRedRowSub.pdf}
\caption{Time series of a 70 pixel radius photometric aperture obtained during a long dark exposure for detector ALONG=A5. The long dark exposure of 108 frames is used to construct 54 pairs of reads that are subtracted, in analogy with a series of integrations.
{\it Top:}  Example read pair subtraction with an overlay of the apertures for 3 different reduction steps from "Raw" data using no reference pixel correction, "Reduced Data" using reference pixels and "ReducedRowColSub" using reference pixels and background subtraction for each row and each column.
{\it Bottom} The time series for each circular aperture with background subtraction using a circular annulus.
The most important step for reducing the noise in the time series is row-by-row subtraction to remove the 1/f noise.
}\label{fig:longDarkPhot}
\end{figure*}

What is most useful for exoplanet transit measurements is the photometric stability within an aperture or spectrum.
We therefore perform aperture photometry on the dark frame as if it were a weak lens image with a 70 pixel radius circular source aperture and a 75 to 100 pixel background annulus illustrated in Figure \ref{fig:longDarkPhot}.
Also, Figure \ref{fig:WLP8PSF} shows an example weak lens PSF with this aperture.
The background subtraction will largely remove the offsets affecting the whole 512$\times$2048 amplifier.
However, the correlated 1/f noise along the X direction will not be subtracted efficiently by an annulus.
Instead, a row-by-row line fit to the background would be more effective.

For a read pair subtraction, the ``CDS'' (correlated double sample) read noise is 13.3 e$^-$ or 7.2 DN for the A5 detector studied here.
If we assume each pixel's read noise is uncorrelated with its neighbors, the total noise within the source aperture is then $\sqrt{N_{px}} \times RN $ where $N_{px}$ is the number of pixels and $RN$ is the read noise of a single pixel.
If we also assume that the background pixels are uncorrelated with each other and the source pixels, then the total noise in background-subtracted photometry is 1310 DN.

This theoretical read noise should be compared to the photon noise expected inside an aperture during photometric stability time series applications.
The number of DN collected within a 70 pixel radius aperture for WLP8 data in our CV3 detector stability test with 2 groups in RAPID (ie a correlated double sample) was $S_{tot}=3.9\times 10^7$ DN during the integration.
The peak pixel counts in this integration were 30,000 DN or about 60\% well depth for the A1 detector.
The Poisson shot noise for photons is $N_{phot} = \sqrt{S_{tot}} / \sqrt{G} = 4,400$ DN, where $G$ is the gain ($G \approx 2.0$ for the SW detectors).
In terms of fractional noise, the photon shot noise should be 113 ppm and the read noise should be 33 ppm.
Therefore, the read noise is expected to be 1/4 the shot noise for 2 group RAPID read mode (ie CDS).
When averaging together 111 full frames over 60 minutes (an order-of-magnitude transit duration for most planets), the photon noise would contribute 11 ppm and the read noise would contribute 3 ppm.



In our experimental time series, however, the standard deviation in the time series in these 3 apertures is 22,000 to 25,000 DN (without any attempt to correct for 1/f noise using reference pixels or otherwise).
If uncorrected, this extra read noise will contribute 560 to 630 ppm to the time series per read pair!
This would be a very inefficient observations if the read noise dominated the $\sim$113 ppm photon noise.
As can be seen in Figure \ref{fig:longDarkPhot}, there are significant correlations along the fast read (X) direction of the detector due to 1/f noise.

\begin{deluxetable*}{ccc}[b!]
\tablecaption{Summary of Noise Sources in Weak Lens +8 Aperture Dark Test}\label{tab:noiseSummaryWLP8}
\tabletypesize{\footnotesize}
%\tablecolumns{7}
%\tablenum{2}
\tablewidth{0pt}
\tablehead{
\colhead{Noise Description} &
\colhead{$\sigma$} &
\colhead{$\sigma$} \\
\colhead{} &
\colhead{DN} &
\colhead{ppm}\\
}
\startdata
Ideal Photon Noise & 4,400 & 113 \\
Ideal Uncorrelated Read Noise & 1,300 & 33 \\
\hline
Measured Read Noise (no correction) & 23,500 & 600 \\
Measured Read Noise (reference pixel correction) & 10,800 & 280 \\
Measured Read Noise (reference channel subtraction) & 9,700 & 250 \\
Measured Read Noise (row-by-row median subtraction) & 6,000 & 150 \\
Measured Read Noise (20 Component PCA subtraction) & 5,400 & 140 \\
Measured Read Noise (smoothed 40 px kernel) & 900 & 23 \\
\enddata
\tablecomments{The theoretical and measured noise sources in one integration for a Weak Lens +8 Time Series.
The noise is the sum within a 70 pixel radius source aperture and 75 to 100 pixel background aperture (in DN) or relative to the photon signal (in ppm).
The integration is assumed to be a bright source with 2 groups of 1 integration each up the ramp (Correlated Double Sampling or CDS mode).}
\end{deluxetable*}

These large correlations in read noise can be reduced with the side reference pixels, which are designed to track electronic noise without the presence of background photon noise.
We use the \texttt{ncdhas} pipeline to apply reference pixel correction, which dramatically reduces the amplifier offsets and also reduces 1/f noise correlations along the fast read (X) direction.
The time series for the 3 apertures considered drops down to 10,000 DN to 11,500 DN, but still falls short of the expectation of 1,310 DN if the read noise is uncorrelated.

Since these 8 reference pixels in each row are not sufficient to remove all 1/f noise in each row, so we perform another correction step:
We find the median of each group up the ramp along the fast read (X) direction and subtract this from every row.
This amounts to a row-by-row background subtraction.

In the latter case, the standard deviation of the time series drops to 6000 DN, but is still larger than the expectation of 1310 DN of all noise were uncorrelated.
The A3 detector used for weak lens photometry has reduced 1/f noise.
The standard deviation of the raw time series is 15,000 compared with the 22,000 from the A5 (ALONG) detector.

\subsubsection{Smoothed Kernel Subtraction}

We also consider more effective ways to reduce read noise than the median row subtraction discussed in Section \ref{sec:longDarks}.
We start by applying a smoothing kernel to each group (1 read for RAPID mode) up the ramp of the raw data.
The kernel is a uniform normalized kernel 1 pixel in height and 40 pixels along a row.
This enables subtraction of 1/f noise that has length scales shorter than the row length (ie timescales shorter than 5 milliseconds, which the median subtraction cannot remove.

This smoothing kernel makes a dramatic improvement to the final weak lens aperture time series, as shown in Figure \ref{fig:longDarkWithRowKernel}.
The standard deviation in the time series drops from the best previous method (row and column median subtraction) by a factor of nearly 10!
For the row and column median subtraction applied to the red files, the standard deviation was 6150, 5600 and 6,600 DN for apertures 0, 1 and 2.
For the same apertures, the standard deviation in the time series drops to 650, 890 and 570 DN.

\begin{figure*}[!hbtp]
\centering
\includegraphics[width=.37\columnwidth]{ap_labels_row_by_row_group_sub_aps_1_NRCALONG-DARK-7235074213_1}
\includegraphics[width=.37\columnwidth]{ap_labels_darkRedsmoothedRowKernel_NRCALONG-DARK-7235074213_1.pdf}
\includegraphics[width=.37\columnwidth]{zoom_long_dark_row_by_row_group_sub.pdf}
\includegraphics[width=.37\columnwidth]{zoom_long_dark_darkRedsmoothedRowKernel.pdf}
\caption{The same apertures as Figure \ref{fig:longDarkPhot}, but with subtraction by a smoothed image where a 40 pixel wide and 1 pixel tall averaging kernel was applied along the fast read direction of every group.
Subtraction by the smoothed image has a dramatic effect on reducing correlated 1/f read noise from $\sim$6,000 DN to $\sim$700 DN in the aperture.
}\label{fig:longDarkWithRowKernel}
\end{figure*}


\subsubsection{Reference Channel Subtraction}
We note that the 4 channels can sometimes exhibit the same 1/f correlation behavior, as shown in Figure \ref{fig:darkPixelTimeSeries}.
Another option to remove 1/f noise is to subtract the illuminated amplifier with a target in it by unilluminated channels.
We attempt to subtract an amplifier 0's time series from all amplifiers.
Of course, this means that this amplifier is perfectly subtracted and no noise or signal remains in this 1/4 of the image.
The subtraction step should increase the read noise on an individual pixel by $\sim \sqrt{2}$ because the errors from the subtracted pixel should add in quadrature to the original pixels.
This increase in read noise on a per-pixel level might be acceptable if it decreases the correlated errors along the rows.
On the other hand, if the each amplifier exhibits its own individual correlated noise behavior, the reference amplifier will fail to subtract out all correlated noise and may introduce more noise.

The net effect of this method is to increase the noise over a row-by-row median subtraction.
The standard deviations in the time series range from 8,500 DN to 10,800 DN for this reference amplifier method as opposed to the 5,500 DN to 6,600 DN for a row-by-row median.
This means the the correlations within an amplifier are not efficiently subtracted and may be propagated to the remaining amplifiers.

\begin{figure*}[!hbtp]
\centering
\includegraphics[width=.32\columnwidth]{pixeltime_series_0.pdf}
\includegraphics[width=.32\columnwidth]{pixeltime_series_1.pdf}
\includegraphics[width=.32\columnwidth]{all_amp_periodograms.pdf}
\caption{The time series for all 4 amplifiers share common-mode behaviors in some cases (left) but not always (middle).
This is visible in a short segment of the time series of all pixels for an example group up the ramp 53 in a dark ramp where the bias has already been subtracted.
The left panel shows the beginning of the time series, which happens to be dominated by common-mode read noise.
The middle panel shows another part of the time series where Amplifier 3 has its own individual correlated-noise pattern.
The right panel shows the periodogram of each amplifier's time series over the first 10$^5$ clock cycles (0.1) seconds, which exhibits no consistent peaks.
}\label{fig:darkPixelTimeSeries}
\end{figure*}

\subsubsection{PCA Correction}
Another approach to treating the 1/f noise is dimensionality reduction with Principal Component Analysis.
Here, we assume that each column is a random variable that is correlated with the rest of the columns.
Each row represents a different noise iteration.
We first mask all bad pixels by setting all pixels with $| f | > 200$ DN.
The first 10 principal components are plotted in Figure \ref{fig:pcaEigenvectors} for two example groups (54 and 55).
These first 10 principal components explain 10\% of the total variance in each group up the ramp.

\begin{figure*}[!hbtp]
\centering
\includegraphics[width=.4\columnwidth]{pca_dark_amp_all_grp54.pdf}
\includegraphics[width=.4\columnwidth]{pca_dark_amp_all_grp55.pdf}
\caption{Principal component eigenvectors identified from group 54 (left) and group 55 (right).
}\label{fig:pcaEigenvectors}
\end{figure*}

We begin by calculating new principal components and eigenvectors for each group up the ramp and then subtracting the first 10 principal component eigenvectors multiplied by their respective principal components by each image.
It is important to mask the bad pixels before calculating the principal components so they are not driven by errant bad pixels.
We also calculate the PCA eigenvectors after the bias has been subtracted and the reference pixel correction was applied.
The PCA subtraction is applied to each group up the ramp and then these groups are subtracted in pairs to create a time series.
The final time series has and measured standard deviation of 5,000 to 5,700 DN within the background-subtracted aperture.
This is comparable to the row and column subtraction standard deviation of 5,500 to 6,500 DN.
Adding 10 more principal components for a total of 20 eigenvectors only reduces the time series noise to 5,100 to 5,400 for the 3 apertures.

\clearpage

\subsubsection{Grism Aperture}
\begin{figure*}[!hbtp]
\centering
\includegraphics[width=.32\columnwidth]{aps_3_NRCNRCALONG-DARK-72350742131_1_485_SE_2017-08-23T16h49m51_int_015_014.pdf}
\includegraphics[width=.32\columnwidth]{aps_0_NRCNRCALONG-DARK-72350742131_1_485_SE_2017-08-23T16h49m51_red_int_015_014.pdf}
\includegraphics[width=.32\columnwidth]{aps_1_NRCNRCALONG-DARK-72350742131_1_485_SE_2017-08-23T16h49m51_red_int_015_014_rowColSub.pdf}
\includegraphics[width=.32\columnwidth]{long_dark_darkGrismApA5raw.pdf}
\includegraphics[width=.32\columnwidth]{long_dark_darkGrismApA5red.pdf}
\includegraphics[width=.32\columnwidth]{long_dark_darkGrismApA5redRowCol.pdf}
\caption{Time series of a grism photometric broadband aperture during a long dark exposure for detector ALONG=A5. As in \ref{fig:longDarkPhot}, the long dark exposure of 108 frames is used to construct 54 pairs of reads that are subtracted, in analogy with a series of integrations.
The photometric aperture is made to represent a similar one to the F322W2 grism trace.
}\label{fig:longDarkGrism}
\end{figure*}


%% See the notebooks wlp8_psf_image.ipynb and noise_sources_wlp8.ipynb for calculations
\subsubsection{The impact of correlated read noise on time series}
We consider here some example science impacts of correlated read noise in these photometric apertures.
We will evaluate the performance of a WEAK LENS +8 mode photometric time series, as will be common with NIRCam grism time series observations.
For a pessimistic scenario, we assume a 15,000 DN read noise contributes to every pair of reads for the 70 pixel radius aperture (and 75 to 100 pixel background subtraction annulus).
If the read noise is tolerable under this scenario, then it should have a minimal impact to photometric time series.
If it is a problem, we will discuss methods of mitigation such as the row-by-row subtraction above.

We also assume that each read/frame is statistically independent from the other.
Therefore, the uncorrected read noise per integration is:
\begin{equation}
RN_{ap,70,75,100} = \frac{21,200 DN}{ \sqrt{N_{frames}}} \approx \frac{46,000 e^-}{ \sqrt{N_{frames}}},
\end{equation}
where $N_{frames}$ is the number of frames in an integration and a gain of 2.17 $e^-$/DN is assumed for the A3 detector.
If there are 2 frames in an integration, we recover 15,000 DN for the case of a read pair subtraction on the A3 detector.
\textcolor{red}{\bf Make sure I scaled this correctly both for the read pair subtraction and a slope fit. Also empirically verify this with a slope fit to multiple images.}
For an integration that fills to a peak of $\sim$ 60\% well depth or 35,000 DN, the integrated DN counts for a WEAK LENS +8 PSF would be 10$^8$ DN, having a photon uncertainty of 9,100 DN or 110 ppm.
Therefore, the correlated read noise, if left uncorrected, will have a highly significant ($>50\%$) impact on an integration of 5 or fewer frames that fill to 60\% well depth.
As an example, a $K$=8.5 magnitude star observed in full frame imaging WEAK LENS +8 imaging mode with 6 RAPID groups up the ramp sampling and the F210M filter would have a photon noise equal to the (uncorrected) read noise and fill the detector to 54\% well depth.


\textcolor{red}{\bf Include more calculations on how the individual frame read noise would impact many frames in a time series of a bright target vs a faint target.}



\subsection{Cryogenic Vacuum (CV) Stability Tests}

During the Cyrogenic Vacuum Test 3, the Integrated Science Instrument Module (ISIM) was placed in a vacuum chamber with the Optical telescope element simulator (OSIM).
The ISIM has all the instruments with their internal optics and detectors but no telescope mirrors.
The OSIM simulates the PSF expected when a star or other point source illuminates the JWST instruments.
While other tests described in this work illuminated the science instrument detectors in different ways, the point spread function most closely resembled the JWST flight point spread function.

CV3 included a series of exposures over nearly 24 hours with an LED lamp in imaging mode to test the fundamental photometric performance.
The LED was a narrowband source at 1.55 $\mu$m, which was used in imaging mode since this would result in a narrow-lined spectrum.
While the spectroscopic modes are expected to be the most scientifically interesting modes for studying exoplanet atmospheres, the broadband lamps needed to produce a spectrum tend to be less stable than LEDs.
Therefore, the test was designed with an LED with the hope of measuring the precision possible with the NIRCam optics and detector.

As with the spatial scanning mode on HST \citep[e.g.][]{mccullough2012spatialScan,deming13}, spreading the light of a source over many pixels can improve the precision of spectroscopy.
This is also accomplished by ground-based observatories by either de-focusing the telescope \citep{southworth2009defocusing} or by employing a beam-shaping diffuser \citep{stefansson2017diffusers}.
By spreading the light source over many pixels, the light curve becomes less sensitive to flat fielding or intra-pixel sensitivity errors with telescope pointing jitter or seeing variations during an observation.
Similarly, NIRCam can employ a weak lens to defocus a star's light.
Figure \ref{fig:WLP8PSF} show the PSF of the +8 wave defocused images during the CV3 stability tests.
The hexagonal PSF is summed with circular aperture with a radius of 70 pixels.
The central peak may used to find a centroid of the star and center the circular aperture.

\begin{figure}[!hbtp]
\centering
\includegraphics[width=.49\columnwidth]{wlp8_psf.pdf}
\caption{Weak Lens +8 PSF during CV3 testing.
The central peak is used for centroiding, while the entire PSF is used for aperture photometry.
A 70 pixel radius aperture (shown here is a black circle) is used for the source aperture.}\label{fig:WLP8PSF}
\end{figure}

A series of detector stability tests was performed using this +8 wave weak lens and 1.55 $\mu$m LED in a few different modes.
A time series of all possible modes is shown in Figure \ref{fig:CV3longTser}.
The full frame modes (FULL1 through FULLQ) used 2 samples of the ramp for integration times of 21.5 seconds whereas the subarrays were tuned to have a similar integration time (21.3 sec for 20 Groups of a 320$\times$320 subarray and 25.1 seconds for 6 groups of a 640$\times$640 subarray).
This ensured that during these 4 different types of subarray/full frame window sizes, the well depths achieved were similar.

It is apparent from Figure \ref{fig:CV3longTser} that the WLP8SUB test with a 320$\times$320 subarray and 20 groups of the ramp has smaller scatter than the FULL1 through FULLQ tests.
After removing a linear fit to the time series (to account for long term trends in the lamp), the standard deviation of the WLP8SUB data is 500 ppm whereas the FULL tests range from 1020 to 1540 ppm.
For the FULL 2048$\times$2048 pixel integrations with 2 groups up the ramp, the count rate of photons is calculated in each pixel by subtracting the two groups (ie two reads for this RAPID mode) from each other and dividing by the group time (time between these two reads for this RAPID mode).
For the WLP8SUB mode, a linear least squares fit is performed to the groups as a function of time and the slope of this line is the count rate on a pixel.
The large number of groups used to fit this line ensures that the read noise is reduced by $\sim 1/\sqrt{N_G}$, where $N_G$ is the number of groups.

There are two explanations for why the WLP8SUB mode shows less scatter than the FULL modes, 
\begin{enumerate}[noitemsep]
	\item The reduction of read noise improves the stability of the signal and\label{item:readNReasonCV3subFull}
	\item That the lamp is variable on short time scales of a frame (10.74 seconds).\label{item:lampReasonCV3subFull}
\end{enumerate}
The read noise of a pixel is $RN_1 \approx$ 15 e$^-$ for the short wavelength arrays\footnote{https://jwst-docs.stsci.edu/display/JTI/NIRCam+Detector+Performance} in this test.
If this is uncorrelated across the 70 pixel aperture source radius, the read noise from two groups  (correlated double sampling) on the total aperture is $RN_{tot} = \sqrt{N_{s,px}} RN_{CDS} = \sqrt{\pi} r_{s,px} RN_{CDS}$ = 124 * 15 e$^-$ = 1861 e$^- \approx 930 $DN.
This amounts to 25 ppm, which is a small compared to the photon noise (113 ppm).
Furthermore, the read noise for 20 groups is $RN_{20} \approx 6e^-$ would ammount to a 10 ppm, so a difference from the CDS readout of only 15 ppm. {\textbf{\textcolor{red}{(Note, here I have used the effectice read noise for 90 groups on JDox, but should corrected to 20 groups)}}
This is dramatically smaller than the observed difference in standard deviation (500 ppm in subarray with 20 groups and 1020-1540 ppm in full frame in 2 groups).
However, if there are correlated components to the read noise in this large aperture, the read noise could be higher.

Under scenario \ref{item:lampReasonCV3subFull}, the short timescale variations of efficiently averaged out by sampling the reads multiple times up the ramp.

\begin{figure}[!hbtp]
\centering
\includegraphics[width=.99\columnwidth]{plot_mmm.pdf}
\caption{Time Series plot through the CV3 photometric stability campaign.
All test shown above were using the Weak Lens +8 mode, with a PSF as shown in Figure \ref{fig:WLP8PSF}.
13 tests (FULL1 through FULLQ) are all in full frame mode, while WLP8SUB used a 320$\times$320 subarray and WLP8SUB640 used a 640$\times$640 subarray.
The WLP8SUB showed the smallest standard deviation of any testof any NIRCam mode explored.}\label{fig:CV3longTser}
\end{figure}

\subsection{Lessons Learned from CV3 Tests}\label{sec:CV3Lessons}
Although the light curves from the CV3 stability tests were not able to achieve photon-limited performance to study the precision of transit and eclipse spectroscopy, there were some lessons learned from this data.
We were able to determine the best practices for the pipeline and analysis by running the raw data through the pipeline and changing the steps to produce light curve.
We experimented with the methodology for turning raw detector reads up the ramp into count rates on the detector, the intrapixel capacitance (IPC) correction, flat fielding, reference pixel correction and detector non-linearity.
After the calibrated images are created, we experimented with different types of aperture photometry including different geometries, background calculations and aperture sizes.

After experimenting with these configurations, we find the following pipeline steps are important:
\begin{itemize}
	\item Fitting a line up the ramp gives superior results to subtracting the last and first pairs of read (as expected by reducing the photon noise)
	\item Centering the aperture with the central spot inside the hexagon is important to conserve the flux within the aperture
	\item Row-by-row subtraction (along the detector's fast-read direction) gives superior background subtraction to traditional aperture photometry (using the mean or robust average value in a background annulus)
	\item Aperture sizes for the source at ~70 pixels, outside the hexagon, gave the best precisions
	\item An annulus aperture on the source (including the inner and outer edges of the hexagonal PSF) usually gives the best precision light curves
\end{itemize}

We also found reduction steps that had very little impact on the final photometric precision including
\begin{itemize}
	\item IPC corrections ($\lesssim 0.01$ ppm)
	\item Flat-fielding ($\lesssim 0.3$ ppm)
\end{itemize}
We summarize the lessons from the tests here.

\subsection{OTIS Tests}

A series of NIRCam stability tests were carried out while the entire JWST telescope system was placed in a tall vacuum chamber at NASA Johnson.
However, there is no feasible way to place a point source lamp at infinity to simulate the imaging of astrophysical objects onto a JWST instrument, so an alternate lamp setup is used.
During these tests, there were two configurations 1) the ``half pass" tests where lamp sources within the optical system and 2) ``pass and a half" tests where the a lamp source within the optical system bounces light off the secondary, primary, flat mirrors mounted on the chamber's ceiling and then through the entire telescope and instrument chain.

The OTIS testing included a time series mode similar to the grism time series template used in orbit for transiting planets.
Figure \ref{fig:otisPSFs} shows the point spread functions for the OTIS tests that simultaneously illuminated the short wavelength and long wavelength channels.
The ``O9'' supercontinuum illuminated the detector over a wide wavelength range from 2.5 to 4.1$\mu$m.
A small fraction of the light is emitted at wavelengths shorter than 2.5 $\mu$m, which is sufficient to illuminate the short wavelength detector to peak counts of $\sim 260$DN per exposure.
Meanwhile, the peak of the grism is $\sim$ 26,000 DN.
This lamp configuration was a pass and a half test, which suffered from the vibrations and oscillations of the suspension system, which can be perturbed by motors such as vacuum pumps.

The point spread function through the weak lens and optical setup create a very unusual point spread function for the short wavelength channel, shown in Figure \ref{fig:otisPSFs} that has multiple clumps of emission.
We perform aperture photometry on a knot of emission at (1090 px, 230 px) for the short wavelength detector.
This knot is concentrated enough to fit with a 2D Gaussian and centroid this Gaussian to ~0.2 pixels, which can be used to track the motions of the PSF due to the oscillations
For the long wavelength detector, we initially extract a box aperture to create a broadband light curve.


\begin{figure*}
\gridline{\fig{wlp8_otis_psf.pdf}{0.7\textwidth}{WLP8 Image} }
\gridline{\fig{{grism_otis_psf.pdf}}{0.7\textwidth}{Grism Image}}
\caption{Median images of the short wavelength (SW) weak lens mode and long wavelength (LW) grism spectroscopy that were used simultaneously in a grism time series stability test.
Due to the OTIS optics setup, the short wavelength WLP8 image does not nearly represent the hexagonal PSF from CV3 (Figure \ref{fig:WLP8PSF}) or flight.
}\label{fig:otisPSFs}
\end{figure*}

\begin{figure*}
%\gridline{\fig{tser_LW_OTIS.pdf}{0.99\textwidth}{WLP8 Image} }
\gridline{\fig{tser_LW_OTIS.pdf}{0.6\textwidth}{Grism Time Series}}
\caption{Time series of the simultaneous SW/LW Time Series Stability test
\textcolor{red}{Thing to try: did I try a circular aperture centered on the knot? try to use centroid from SW to adjust the position of the source extraction box}
}\label{fig:otisLWSWtser}
\end{figure*}


As with the CV3 tests in Section \ref{sec:CV3Lessons} and Figure \ref{fig:CV3longTser}, we found the highest precision lightcurves when using subarray read configurations.
Figure \ref{fig:OTISsubarrays} shows the many subarrays explored in a set of time series.
As the number of reads up the ramp becomes larger and larger, the standard deviation increases.
This would be expected if the read noise contributed significantly to the standard deviation so that ramp fitting decreases the read noise.
The perplexing thing is, though, that this is even true if we throw away all the intermediate reads.
That is, with all readout modes, count rate (ie. slope image is approximated by the last subtracted by the first read.
Here the number of reads in between these is irrelevant.
A possible way to explain the superior standard deviation for the smallest 2048$\times$64 subarray is not that more reads increase precision but rather the small subarray has a smaller contribution from spatially correlated noise.
When larger frames are read out, for example, the 1/f noise becomes increasingly uncorrelated from read to read.\textcolor{red}{Wrote this when very tired. Is this correct and logical?}

Note that the integration times are different for these 4 tests. They are 39.9, 39.9, 32.3, 31.8 and 32.2 for the 64-STRIPE, 128-STRIPE, 256-STRIPE, 256-WIND and FULLFRAME respectively. Perhaps the lamp has oscillations on a timescale that are averaged out efficiently at 39.9 seconds but not 32.3 seconds?

\begin{figure}[!hbtp]
\centering
\includegraphics[width=.5\columnwidth]{subarray_config_tser.pdf}
\caption{A variety of subarrays in the OTIS stability tests were read out with similar integration times (32 to 39 seconds) to compare relative precision of the read modes. 
The time series are ordered vertically in increasing frame duration (and decreasing number of groups up the ramp) toward the bottom.
With this very crude box extraction (and no background subtraction), the standard deviation decreases as the number of groups up the ramp becomes more numerous.
This is likely related to correlated 1/f noise in the horizontal direction that is averaged out by more reads up the ramp.}\label{fig:OTISsubarrays}
\end{figure}

\begin{figure}[!hbtp]
\centering
\includegraphics[width=.5\columnwidth]{fullframe_background_sub_compare_tser_tools.pdf}
\caption{An experiment with circular background-subtracted photometry shows dramatically improved precisions over raw aperture sums in Figure \ref{fig:OTISsubarrays}.
Additionally, a median subtraction of all background pixels per row improves the precision.}\label{fig:OTISSubConfigCircularAp}
\end{figure}

\begin{figure}[!hbtp]
\centering
\includegraphics[width=0.48\columnwidth]{apertures_for_NRCN83-128-STRIPE-72.pdf}
\includegraphics[width=0.48\columnwidth]{apertures_for_NRCN83-256-STRIPE-72.pdf}
\includegraphics[width=.48\columnwidth]{subarray_config_tser_backsub_rect.pdf}
\caption{With rect backub.
\textcolor{red}{To do: understand better why the vertically offset box is better? Also, show the aperture as a separate figure.}}\label{fig:OTISRectBacksub}
\end{figure}

\subsection{GL Dewar Tests}

The GL dewar, located at the University of Arizona, is a laboratory setup for assessing NIRCam flight and spare detectors.
The key difference with the Asic dewar discussed below is that the detector readout electronics are completely different, using a Leach system that is operated outside of the dewar as opposed to the Asic, which is lives right next to the detectors.
The GL dewar detector stability tests are helpful in isolating noise sources that arise from the detectors versus the readout electronics.

An LED illumination source is pointed at the detectors and the arrays are read out in subarray mode.
Figure \ref{fig:GLtSeries} shows the detector subarray illumination as well as two apertures that were used for time series photometry.
The first handful of frames show an initial "warmup" period lasting about 9 seconds.
This quickly stabilizes to a 1200 ppm level likely dominated by lamp variations.
If one divides the two apertures by each other, the resulting time series is very flat and stable to with 260 ppm.
When binning the data into 2.5 minute increments, the standard deviation is only 51 ppm.


\begin{figure*}
\gridline{\fig{ap_labels_S02illum_GLrun104.pdf}{0.32\textwidth}{Apertures}
		\fig{raw_tser_phot_S02illum1miAll_GLrun104.pdf}{0.32\textwidth}{Raw time series}
		\fig{refcor_phot_S02illum1miAll_GLrun104.pdf}{0.32\textwidth}{Ratio Time Series}}
\caption{The precision of this short-duration GL dewar test was the extremely high at $\sigma=$240 ppm after stabilization, if we remove absolute changes likely due to the lamp.
}\label{fig:GLtSeries}
\end{figure*}

\begin{figure*}
\gridline{\fig{ap_labels_S02illum1miAllBacksub_GLrun104.pdf}{0.32\textwidth}{Apertures}
		\fig{raw_tser_phot_S02illum1miAllBacksub_GLrun104.pdf}{0.32\textwidth}{Raw time series}
		\fig{refcor_phot_S02illum1miAllBacksub_GLrun104.pdf}{0.32\textwidth}{Ratio Time Series}}
\caption{The same time series as Figure \ref{fig:GLtSeries} but with a rectangular background subtraction box.
\textcolor{red}{To do: understand why the ratio of the two box time series has a larger standard deviation than in Figure \ref{fig:GLtSeries}. Is it just the poor SN from a small background box? Also, separately, compare two different detectors.}
}\label{fig:GLtSeriesBacksub}
\end{figure*}

\subsection{Asic Dewar Tests}

%% If you wish to include an acknowledgments section in your paper,
%% separate it off from the body of the text using the \acknowledgments
%% command.
\acknowledgments

\section{Expected In-Flight Performance with JWST}

\section*{acknowledgements}
MCMC fitting makes use of \texttt{emcee} \citep{foreman-mackey2013emcee} and the covariance plot was made with \texttt{corner.py} \citep{foremanCorner}.
Funding for the E Schlawin is provided by NASA Goddard Spaceflight Center.
This research has made
use of the Exoplanet Orbit Database and the Exoplanet Data Explorer at \url{http://exoplanets.org}; SIMBAD database, operated at CDS, Strasbourg,
France; NASA's Astrophysics Data System Bibliographic
Services; the M, L, T, and Y dwarf compendium
housed at \url{http://DwarfArchives.org}; the SpeX Prism
Libraries at \url{http://www.browndwarfs.org/spexprism}; and the \texttt{astropy} package \citep{astropy2013}. 
The authors wish to recognize and acknowledge the very significant cultural role and reverence that the summit of Mauna Kea has always had within the indigenous Hawaiian community. We are most fortunate to have the opportunity to conduct observations from this mountain.

%% To help institutions obtain information on the effectiveness of their 
%% telescopes the AAS Journals has created a group of keywords for telescope 
%% facilities.
%
%% Following the acknowledgments section, use the following syntax and the
%% \facility{} or \facilities{} macros to list the keywords of facilities used 
%% in the research for the paper.  Each keyword is check against the master 
%% list during copy editing.  Individual instruments can be provided in 
%% parentheses, after the keyword, but they are not verified.

\vspace{5mm}
\facilities{HST(WFC3), IRTF(SpeX)}

%% Similar to \facility{}, there is the optional \software command to allow 
%% authors a place to specify which programs were used during the creation of 
%% the manusscript. Authors should list each code and include either a
%% citation or url to the code inside ()s when available.

\software{astropy \citep{astropy2013}, 
          \texttt{emcee} \citep{foreman-mackey2013emcee}, 
          \texttt{batman} \citep{kreidberg2015batman},
          \texttt{spiderman} \citep{louden2017spiderman},
          \texttt{pynrc} \url{https://github.com/JarronL/pynrc},
          \texttt{emcee} \citep{foreman-mackey2013emcee}
           }

%% Appendix material should be preceded with a single \appendix command.
%% There should be a \section command for each appendix. Mark appendix
%% subsections with the same markup you use in the main body of the paper.

%% Each Appendix (indicated with \section) will be lettered A, B, C, etc.
%% The equation counter will reset when it encounters the \appendix
%% command and will number appendix equations (A1), (A2), etc. The
%% Figure and Table counter will not reset.

\appendix

%\section{Extra Info}

%% The reference list follows the main body and any appendices.
%% Use LaTeX's thebibliography environment to mark up your reference list.
%% Note \begin{thebibliography} is followed by an empty set of
%% curly braces.  If you forget this, LaTeX will generate the error
%% "Perhaps a missing \item?".
%%
%% thebibliography produces citations in the text using \bibitem-\cite
%% cross-referencing. Each reference is preceded by a
%% \bibitem command that defines in curly braces the KEY that corresponds
%% to the KEY in the \cite commands (see the first section above).
%% Make sure that you provide a unique KEY for every \bibitem or else the
%% paper will not LaTeX. The square brackets should contain
%% the citation text that LaTeX will insert in
%% place of the \cite commands.

%% We have used macros to produce journal name abbreviations.
%% \aastex provides a number of these for the more frequently-cited journals.
%% See the Author Guide for a list of them.

%% Note that the style of the \bibitem labels (in []) is slightly
%% different from previous examples.  The natbib system solves a host
%% of citation expression problems, but it is necessary to clearly
%% delimit the year from the author name used in the citation.
%% See the natbib documentation for more details and options.

\bibliographystyle{apj}
\bibliography{this_biblio}

%% This command is needed to show the entire author+affilation list when
%% the collaboration and author truncation commands are used.  It has to
%% go at the end of the manuscript.
%\allauthors

%% Include this line if you are using the \added, \replaced, \deleted
%% commands to see a summary list of all changes at the end of the article.
%\listofchanges

\end{document}

% End of file `sample61.tex'.
